% To talk:
% - Nespresso, their logistics and how can RFID help
% - grab nespresso example e estrapolar para outros markets
% - CTT, logistica, integração com amazon, how can RFID help
% - *escolas/hospitais: controlo de material disponível
% - Necessity of RFID globaly: optimizar transporte (contentores de transporte maritimo), data analisys for predictions, um tecnologia que unifica APIs de todas as empresas 
% - Benefícios de smart shelves e de que forma se encaixa no environment  

\chapter{Introduction}

\section{Background and Motivation}

The new industrial revolution is centered around data. Yet, the true extension of this reformation might not be fully understood.

From manufacturing to retail, information and communication technologies disrupted the understanding of what could be improved through the supply chain~\footnote{the sequence of processes involved in the production and distribution of a commodity~\cite{OxfordLanguagesGoogle}}. 
To grasp the advantages of \ac{rfid}, we have to understand its place within the the other technologies and how it can empower the relationship between manufacturing, transportation, logistics and retail.

The introduction of information and communication technologies in the manufacturing process, has been the foundation of Industry 4.0, which some people refer has the fourth industrial revolution~\cite{marrWhatIndustryHere}. It expanded the cyber-physical systems \footnote{a system in which a mechanism is controlled or monitored by computer-based algorithms} of the third industrial revolution and introduced \ac{iot}, cloud and cognitive computing.
The rise of wireless communications and the massification of data sensing in the manufacturing processes, allowed these technologies to enhance and optimize production processes, logistics operations and marketing strategies.

Adding \ac{iot} provided the network infrastructure for data transfer. Cloud computing enabled on-demand availability of \ac{it} resources, fast deployment times and more efficient management strategies. Cognitive computing contributed with computer vision, signal processing and tools to analyze and process large amounts of data for operational optimizations and pattern recognition.
These technologies support the modern digital end-to-end system of a business and its manufacturing processes.

\ac{uhfrfid} improves the developments made in \emph{Industry 4.0} by adding digital visibility of physical assets. But visibility inside a company is opaque to trading partners and vice versa. Visibility both inside and outside of the company is desirable. With \ac{epc} that is possible. Inside the company, physical objects can be digitalized, identified and wirelessly tracked. Outside, frontiers between company methods of operations and tools, can be abstracted in a common set of interfaces and standards that allows visibility of products outside company bounds and share of information between trading partners.
This achieves an end-to-end integration of the complete commercialization process, where data from billions of physical items can be shared through the internet, enabling businesses and consumers to identify, locate and engage each item.

The future of \ac{rfid} seems promising. In 2016, 96\% of apparel retailers had plans to deploy tags on their products~\cite{hardgrave2016StateRFID}. The \ac{dod} requires that all contractors must use \ac{uhfrfid} identification since 2005~\cite{DODReleasesFinal}.In 2014, Impinj, Intel, Google and Smartrac teamed up to form RAIN RFID, an organization whose mission is to promote the adoption of \acs{epc} \ac{uhfrfid}. \ac{uhfrfid} can greatly improve transport systems with cost reductions, smaller inventory, faster transportation and routing troubleshooting, lower insurance rates and greater efficiency~\cite{oanaRFIDTechnologyContainers2013}.

Development and standardization of \ac{uhfrfid} enables a symbiotic relation across the chain, connecting production, logistics, retail and client.

\section{Scope}

This dissertation intents to explore the implementation of smart shelve and software for retail inventory management systems following the global supply chain standards. 


% Smart Shevels: where they stand in the RFID chain, what they bring to the market

% Nespresso: present conditions, how smart shelves can help (logistics, supply chain)

% The proposed solution is a system around \textbf{smart shelving}. 

% The structure storing the products contains RFID antennas and readers that detect and read the tags attach to them. Those readers will let the platform know in real time the state of the product in stock.

% This system should handle the registration and verification of arriving stock and manage in real time warehouse products. 

%The product should integrate with the logistics management software used by the company, allowing the a real-time management off all the products, machine learning predictions and control of the product flow.

%The solution must be reliable and cheap to maintain. The initial investment should also be the smallest possible. 

%Nespresso is owned by Nestlé Nespresso S. A., one operational unit of the Nestlé Group, with headquarters in Lausanne, Switzerland~\cite{nespressowebsite}. (...)

%With the growing of the brand, the complexity in the logistics networks starts to compromise the management of the products down in the chain. 

%The categorization and verification of new inventory, inspection of the arrived goods from the transportation company, returns, control and management of stocks, are all attended by manual labour. 
%The manual labour is prone to errors, takes a lot of working time and interfacing with the management software isn't usually efficient.

%\section{Objectives}

%\begin{itemize}
  %\item \textbf{Prevent stock-outs:} get timely replenishment and optimizes in-store sales and management. Logistics companies deliver goods on time and according to delivery requirements;
  %\item \textbf{Reduce time and errors from manual labour:} counts, identification, misplacement and lost or stolen items;
  %\item \textbf{Help customers:} find and engage with the products they want;
  %item \textbf{Control:} who removes or checks out valuable items;
  %\item \textbf{Automatic information and management of stock:} the logistics lines automatically transmits and receives stock information;
  %\item \textbf{Smart physical storage:} automatic identification of goods in the warehouse/shelves, automatic matching of distribution requirements, improves the efficiency of goods storage;
  %\item \textbf{Acquisition technology:} After the goods enter the collection area, the collection equipment automatically identifies multiple items by collecting RFID tags, thereby efficiently completing the goods in and out of the warehouse, ensuring whether the physical and distribution requirements are consistent, and improving the efficiency of goods distribution;
  %\item \textbf{Real-time:} master the distribution of all goods in real time, accurately grasp the inventory situation, optimize the reasonable inventory, and grasp the status and changes of the warehouse environment in real time;
%\end{itemize}

\section{Outline}

%\cleardoublepage