\chapter{EPCglobal Architecture Framework}

\section{Context}

Advances in \ac{uhfrfid} gathered a lot of attention and investment in the beginning of the decade. The technology promised, for years, a disruption in the \ac{scm} which was never delivered. 

Item level identification allows companies to capture product lifecycle information at remarkable levels of detail. \ac{rfid} readers placed through out the supply chain can automatically capture information about tagged objects while they move from manufactured to consumer.
An infrastructure that bridges the gap between the physical and the digital world, providing real-time information about current supply chain operations.
Furthermore, the instant share of information between intervening companies increases supply chain visibility, resulting in reduced uncertainty in operational and tactical supply chain planning.
Stock levels could be precisely controlled and shared with trading partners, which in turn reduces inventory costs and optimizes intra-company operations~\cite{lorenzDiscoveryServicesEPC2011, simchi-leviCadeiasSuprimentosProjeto2003}.

It was an utopia ahead of is time. The technology was not mature enough, the return on investment was not appealing and cloud computing had just started to get traction.

This did not stop the conceptualization and development of architectures capable of delivering the promises we were hoping for.
The architecture that stood relevant through out these uncertain times is the EPCglobal Architecture Framework.
It was created and maintained by GS1, a non profit organization tasked with developing global standards for business communications.
The organization has the experience, resources and influence to make this utopia a reality.
The architecture shows on paper an appealing concept of a network capable of doing amazing thing for the \ac{scm} without restricting businesses \ac{it} architectures implementations.

\subsection{Standardization Efforts}

Standardization of \ac{uhfrfid} for item level tagging and supply chain, by organizations like GS1, provided a common language to identify, capture and share supply chain data, ensuring important information is accessible, accurate and easy to understand~\cite{anonymousStandardsGS12014}.

The first prominent adoption was by the \ac{dod} with a policy released on July 30th 2004. The policy stated that contracts issued for material delivery would require the use of \ac{uhf} tags. The policy was later extended to all commodities and commodities pallets shipped to any \ac{dod} facility~\cite{DoDSuppliersPassive, DODReleasesFinal}.

In 2014, Impinj, Intel, Google and Smartrac, joined forces to create the RAIN RFID alliance after the ratification of GS1's \ac{uhfrfid} Generation 2 version 2 standard in November of 2013. The alliance promotes the universal adoption of GS1's Gen2 \ac{uhfrfid} technologies and cloud computing, where \ac{rfid}-based data can be stored, managed and shared via the Internet~\cite{WhatRAINRFID}.
The alliance fortified the adoption of GS1's standards and traced a common path for the the industry to progress~\cite{TechnologyCompaniesCreate}.

On October 11th 2018 the European Commission published their positive implementation of the upper band for European countries~\cite{302208v030101pPdf}.
It extended the power levels to $2W$ in the lower band and added the requested global band from $915$MHz to $921$MHz with power levels up to $4W$. 
This was the biggest effort by the European Commission to establish a global standardized frequency band for \ac{uhfrfid} supply chain applications.

\subsection{Current Problems} \label{sec:currentproblems}

There are still \ac{rfid} tags that do not conform with the international standards, often presenting proprietary formats and even encoding errors.
These closed practices and struggle for market supremacy around \ac{uhfrfid} creates a problematic situation that prevents conformity through the global logistics market.

Even in global standards, the adoption depends on the company and field of business. Usually one identification standard is already being used, and the migration cost for supporting multi-code integration can be high.

The information around global standards is also limited and hard to get through. It is divided in multiple specifications, identified with number notation and codification nomenclature. 
The ISO standards, in specific, are closed and have to be payed before even see it's contents.
These specifications are extensive and don't provide newcomers a good experience. 
Companies planning to implement \ac{uhfrfid} systems following legitimate global standardization resort to consultants who have a deep knowledge on the standards complexity.

The closed mentality in the area slowed the industry progression. In comparison with the cloud and web industries, where experience and software is shared and open-sourced, \ac{uhfrfid} tends to keep everything closed~\cite{WhatCouldSlow}. The existent freely available software is old, outdated and out of maintenance. Experience from real-world implementations is unavailable making the industry prone to committing recurrent mistakes. This results in high investments in time and money on engineering resources that could be shared among industry leaders.

The positive implementation of the upper band frequency in Europe for global \ac{uhfrfid} supply chain applications is also dependent on the acceptance by each European members. In particular, Germany and Netherlands are not accepting it~\cite{EUUpperBand}. The conflict with existing adopted bands in the countries makes a global homogeneous system a challenge that will need time to be established.

\subsection{GS1 and EPCglobal}

GS1 is a nonprofit organization dedicated to the development and implementation of standards for global supply chain solutions. 
The institution mission is to manage the GS1 System of Standards, create open, global, multi-sector standards fostering good business practices.

GS1 established itself in 2005 from the \ac{ean} International, \ac{ucc} and other local organizations from the United States~\cite{PublicationLEBENSMITTELZEITUNGa}.
The organization took under its umbrella the former \acs{ean}-\acs{ucc} roles subsuming their technologies. From those, worth mentioning: the barcode identification system (from \ac{ean}), \ac{xml} standards, \ac{edi} transaction sets and supply chain solutions~\cite{lahiriRFIDSourcebook2005}.

The new GS1 organization then adopted much more ambitious projects, developing global standards and services for business communication.
From those efforts resulted the network for the synchronization of master data \ac{gdsn}, the \ac{epc} integration for \ac{rfid}, traceability and the upstream integration of the consumer goods industry suppliers and EPCglobal Network.

For RFID Technology to become viable in practice, an infrastructure must exist for processing and communicating EPC data. In meeting the goal of creating a common infrastructure, \ac{mit} announced Auto-ID Release $1.0$ in October 2003. At the same time, \ac{mit} entered into an exclusive licensing agreement with GS1.
In turn, GS1 established a new division called EPCglobal to implement Release $1.0$ and to conduct further development based on industry input. This put forth an initial set of standards that formed the infrastructure for \ac{epc} data. Later, Auto-ID Release $1.0$ became the starting-point for the EPCglobal Network~\cite{GlobalRFIDValue}.

\section{Overview}
%Activities, Standards, Goals

%EPC Uniqueness, Identifiers, Decentralized Implementation, Issuing Organization
%Review text - it is from and ex subsection prior to the revision of the new index

EPCglobal Architecture Framework is a collection of interrelated hardware, software, and data standards that interoperate with shared network services~\cite{GS1EPCglobalArchitecture}.
These services are referred to as \emph{EPCglobal Network}, a computer network used to share \ac{epc} data between trading partners.
They are operated by GS1, its delegates and others to provide automatic, real-time identification and data sharing of items both within and outside of an company~\cite[p. 213]{lahiriRFIDSourcebook2005}.

The framework defines information systems, interface standards and data models. This approach frees the market of \ac{it} systems to create custom business solutions. Manufacturing can have their custom business logic closed and expose production state information to the clients through the \emph{EPCglobal Network} with the \emph{EPCglobal} interface standards.

The existence of these standards promote not only the global adoption of \ac{epc}, but also the exchange of information between business partners.
Even doe the network was design primarily for \ac{rfid} \ac{epc} data sharing, the network does not exclusively runs on \ac{rfid} data carriers. The Network can also be fed \ac{epc} data through data carriers like 1D and 2D barcodes. The interoperability with the barcode was one of the most important considerations used during the planning of the network~\cite{RFIDBarcodeInteroperability}.

\subsection{Activities}

The architecture defines three core activities, all of which have a group of standards and guidelines within the \emph{EPCglobal Architecture Framework}: \emph{Physical Object Exchange}, \emph{Infrastructure for Data Capture} and \emph{Data Sharing}.
These activities are helpful in understanding the organization and scope of the framework but should not be interpreted as extremely rigid~\cite{GS1EPCglobalArchitecture}.

\subsubsection{Physical Object Exchange} 

\emph{Identify} individual products, cases, loads, assets, return items, among others, so they can be tracked individually.
The \textit{End Users} parties in a supply chain that exchange physical objects that are identified with \acp{epc}.
Physical object exchange consists in operations such as shipping, receiving goods, and so on.
For many End Users, the physical objects are trade goods, but this could not be the case.
There are many other uses, like library or asset management applications~\cite{dong-yingliDesignInternetThings2016} that differ from the supply chain trade goods model, but still involve unique identification and tagging of objects. 
The architecture must be designed to ensure that when one \textit{End User} delivers a physical object to another end user, the latter will be able to determine the \ac{epc} of the physical object and interpret it properly~\cite{GS1EPCglobalArchitecture}.

\subsubsection{Infrastructure for Data Capture} 

\emph{Capture data} about the movement of physical assets and creating visibility.
In order to gather \ac{epc} data, each \emph{End User} carries out operations within its environment. That can be the creation of \ac{epc}s for new objects, follow the movements of objects by sensing their \ac{epc}s, and gather that information into systems of record within the organization~\cite{GS1EPCglobalArchitecture}. The \emph{EPCglobal Architecture Framework} defines interface standards for the major infrastructure components required to gather and record \ac{epc} data.

\subsubsection{Data Sharing} 

\emph{Exchange data} with \ac{it} applications and trading partners, to turn visibility into information and action.
\emph{End Users} benefit from the \emph{EPCglobal Architecture Framework} by sharing data with each other, increasing the visibility they have with respect to the movement of physical objects through the supply chain. 
The \emph{EPCglobal Architecture Framework} defines \ac{epc} data sharing standards, which provide a means for end users to share data about \acp{epc} within defined user groups or with the general public, and which also provide access to \emph{EPC Network Services} and other shared services that facilitate this sharing.

% image: \emph{EPCglobal Architecture Framework} activities~\cite{Architecture6framework20140414Pdf}


\subsection{Standards}

% This standard defines EPC tag data formats for Generation 2 tags. It defines how the EPC is encoded on the tag and how it is encoded for use in the information systems layers of the EPC Systems Network. The standard includes specific encoding schemes for EPC General Identifier (GID).

Specifications on \ac{epc} encoding are defined under the \ac{tds}~\cite{EPCTagData} in conjunction with the encoding of \acp{epc} and data in \ac{gen2} \ac{rfid} tags.

\begin{table}[]
    \begin{adjustwidth}{-0.09\textwidth}{0em}
    \begin{tabular}{|l|l|l|}
    \hline
    \textbf{Activity}                                                                       & \textbf{Standard}                                                                                                                                           & \textbf{Status}                                                                                                                                                                                                                \\ \hline
    Object Exchange                                                       & \emph{\acs{uhf} Gen 2 Tag Air Interface}                                                                                                          & Ratified Jul 2018 v2.1.0~\cite{UHFGen2Tag}                                                                                                                                                                        \\ \cline{2-3} 
                                                                                            & \acs{hf} Class 1 Tag Air Interface                                                                                                         & Ratified Sep 2011 v2.0.3~\cite{HFClassTag}                                                                                                                                                                        \\ \cline{2-3} 
                                                                                            & \emph{\acs{epc} Tag Data Standard (\acs{tds})}                                                                                               & Ratified Nov 2019 v1.13~\cite{EPCTagData}                                                                                                                                               \\ \cline{1-1}
    \begin{tabular}[c]{@{}l@{}}Data Capture \\ Infrastructure\end{tabular} &                                                                                                                                                             &                                                                                                                                                                                                                                \\ \cline{2-3} 
                                                                                            & \emph{Low Level Reader Protocol (\acs{llrp})}                                                                                                     & Ratified Oct 2010 v1.1~\cite{LowLevelReader}                                                                                                                                                             \\ \cline{2-3} 
                                                                                            & Reader Management (\acs{rm})                                                                                                               & Ratified May 2007 v1.0.1~\cite{ReaderManagementRM}                                                                                                                                                       \\ \cline{2-3} 
                                                                                            & \begin{tabular}[c]{@{}l@{}}Discovery, Configuration, and \\ Initialization (\acs{dci}) for Reader Operations\end{tabular}                  & Ratified Jun 2009 v1.0~\cite{DiscoveryConfigurationInitialization}                                                                                                                                       \\ \cline{2-3} 
                                                                                            & \emph{Tag Data Translation (\acs{tdt})}                                                                                                           & Ratified Oct 2011 v.1.6~\cite{tdtTagDataTranslation}                                                                                                                                                     \\ \cline{2-3} 
                                                                                            & \emph{Application Level Events (\acs{tds})}                                                                                                       & Ratified March 2009 v1.1.1~\cite{ApplicationLevelEvents, ApplicationLevelEventsa}                                                                                                                        \\ \cline{2-3} 
                                                                                            & \emph{\acs{epcis} Capture Interface}                                                                                                              & \acs{epcis} Ratified                                                                                                                                                                                                                 \\ \cline{2-3} 
                                                                                            & \begin{tabular}[c]{@{}l@{}}\emph{\acs{epcis} Data Standard}\\ \emph{Core Business Vocabulary (\acs{cbv})}\end{tabular} & \begin{tabular}[c]{@{}l@{}}\acs{epcis} Ratified Sep 2016 v1.2~\cite{InformationServicesEPCIS}\\ \acs{cbv} Ratified Oct 2017 v1.2.2~\cite{CoreBusinessVocabulary}\end{tabular} \\ \cline{1-1}
    Data Sharing                                                           &                                                                                                                                                             &                                                                                                                                                                                                                                \\ \cline{2-3} 
                                                                                            & \emph{\acs{epcis} Query Interface}                                                                                                                & \acs{epcis} Ratified Sep 2016 v1.2~\cite{InformationServicesEPCIS}                                                                                                                                             \\ \cline{2-3} 
                                                                                            & Pedigree Standard                                                                                                                                           & Ratified Jan 2007 v1.0~\cite{PedigreeStandardV1}                                                                                                                                                         \\ \cline{2-3} 
                                                                                            & EPCglobal Certificate Profile                                                                                                                               & Ratified Jun 2010 v2.0~\cite{EPCglobalCertificateProfile}                                                                                                                                                \\ \cline{2-3} 
                                                                                            & Object Name Service (\acs{ons})                                                                                                            & Ratified Jan 2013 v2.0.1 ~\cite{deanGS1ObjectName}                                                                                                                                                       \\ \cline{2-3} 
                                                                                            & Global Data Synchronisation Network (\acs{gdsn})                                                                                           & Ratified Nov 2020 v3.1.14~\cite{joe.horwoodGDSNStandardsMaintenance}                                                                                                                                     \\ \cline{2-3} 
                                                                                            & \begin{tabular}[c]{@{}l@{}}Lightweigh Verificationt Messaging \\ Standard\end{tabular}                                                                      & Ratified Jul 2019 v1.1~\cite{david.buckleyGS1LightweightVerification2019}                                                                                                                                               \\ \cline{2-3} 
                                                                                            & GS1 EDI~\cite{anonymousGS1ElectronicData2014}: \acs{xml} standards                                                   & Ratified Nov 2019 v3.4.1~\cite{david.buckleyGS1XMLStandards2019}                                                                                                                                         \\ \cline{2-3} 
                                                                                            & Discovery Services                                                                                                                                          & In Development                                                                                                                                                                                                                 \\ \hline
    \end{tabular}
    \end{adjustwidth}
    \caption{Standards within the EPCglobal Architecture Framework} 
    \label{tab:standards}
\end{table}

\begin{figure}[!ht]
    \centering
    \includegraphics[width=\textwidth]{./figs/02-state-of-the-art/architecturer_structure.pdf}
    \caption{EPCglobal Architecture Framework and EPC Structures Used at Each Level~\cite{EPCTagData}} 
    \label{fig:archstructure}
\end{figure}

\section{Generation 2 \ac{uhf} Tag}

A \ac{gen2} \ac{rfid} tag is a passive \ac{rfid} tag that conforms to the \ac{epc} Class-1 Generation-2 \ac{uhf} \ac{rfid} Protocol for Communications at $860$ MHz - $960$ MHz standard, the ISO/IEC 18000-6:2013 standard (Type C), or related standards currently under development.
The \ac{epc} Generation 2 Air Interface Protocol defines the physical and logical requirements for \ac{rfid} readers and passive tags operating in the \ac{uhf} band~\cite{UHFGen2Tag}.
In particular, we are interested in Class 1 \ac{uhf} tags. \ac{c1g2} tags operate in the \ac{uhf} or \ac{hf}~\cite{HFClassTag} bands, in \ac{worm} type \ac{rfid} systems like \ac{scm} and logistics.
They are cheap, robust, support cryptographic authentication for anti-counterfeiting, functions for traceability protection and most important, are conformable with the ISO/IEC 18000-6 Type C air protocol, conjugating the two most prominent standards in \ac{uhf}.

It is important to address the \ac{c1g2} and ISO/IEC 18000-6 conformability. Despite the interrogation commands and logical memory map being the same, the standards differ in the encoding of data. 
GS1 and ISO have different formats and encoding rules to represent \acp{id}.
Systems that want to support both encoding types have to implement interoperability between them~\cite{mizutaniMulticodePortableRFID2016a}.
In this dissertation we focus on the \emph{EPCGlobal Architecture Framework}, so ISO encodings will not be covered.

\subsection{Memory}

Figure~\ref{fig:logicalmemorymap} depicts the logical memory layout in \ac{epc} Gen2 tags. It has four banks of non-volatile memory: \emph{Reserved Memory}, \emph{\ac{epc} Memory}, \emph{\ac{tid} Memory} and \emph{User Memory}. Banks can be accessed by multiples of 16 bits words.

\begin{figure}[!ht]
    \centering
    \includegraphics[width=\textwidth]{./figs/02-state-of-the-art/logicmemorymap.pdf}
    \caption{Logical memory map of EPCGlobal \ac{gen2} \ac{uhf} tags~\cite{UHFGen2Tag}} 
    \label{fig:logicalmemorymap}
\end{figure}

\subsubsection{Reserved Memory (Bank $00$)}

Reserved Memory (Bank $00$) holds the Access and Kill passwords, if implemented by the manufacturer of the Tag. 
The \textbf{Kill password} is a 32-bit value used in the \textit{Kill Command} to render a Tag non-responsive thereafter. The \textit{Kill Command} only executes if the password has been set to a value different from the default all-zero password. 
The \textbf{Access password} is a 32-bit value which allows to transition the Tag in to the Secure state. In the Secure state we can execute all Access commands, including writing to locked blocks. The default password is all zeros and must be changed if access protection is required~\cite{RFIDEPCGen2,UHFGen2Tag}.

\subsubsection{\ac{epc} Memory (Bank $01$)}

\ac{epc} Memory (Bank $01$) contains a 16-bit \ac{crc} for error correction, a 16-bit \ac{pc} and a \ac{epc}.

The \ac{epc} is the \ac{id} in which the \emph{EPCGlobal Framework} revolves around to identify every item in the world. It has the characteristic of being extensible, being that depending on the application requirements it can be between 96 bits and the maximum \ac{epc} size supported by the tag (some tags can have up to 496 bits). Important to refer, tag cost rises with the amount of \ac{epc} bits. Companies planning on implementing \ac{epc} must devise a serialization plan. This thematic will be discussed further in this dissertation.
Its encoding is not defined in the \ac{gen2} air protocol standard, but in the \emph{\ac{epc} Tag Data Standard} which will be further discussed with detail on section~\ref{sec:epc}.

The standard also defines \ac{xpc} words, XPC\_W1 and XPC\_W2 respectively. These words are optional and not commonly implemented, depending on tag manufacturer to do so. In fact, the tags used in this dissertation use \textit{Impinj's Monza} chips, widely used in all kinds of tag applications around the globe, which don't implement \ac{xpc} words. These words contain bit indicators for settings like hazardous materials and intractability to say a few.

The \ac{pc} word contains metadata used to interpret the \ac{epc}. A more detailed representation of the \ac{pc} can be seen in table~\ref{tab:pcbits}.
From $10_h-14_h$, 4 bits with the length of the \ac{epc}/\ac{id} in words, at $15_h$ a toggle bit called \ac{umi} which indicates if \emph{User Bank $11$} has any data in it, and at $16_h$ a toggle bit to indicate if the tag has extended \ac{pc}.
To distinguish an GS1 from an ISO standard \ac{id}, the most reasonable way is to look at bit $17_h$, which contains a toggle bit to indicate whether the \ac{id} is GS1 or non-GS1 family.
From bit $18_h$ to $1F_h$ is reserved for future use under the \emph{EPCGlobal} specification. Under some circumstances GS1 EPCglobal may permit another standards body or organization to use one or more of these RFU values for standardization purposes. Also to note, on ISO 18000-6, these bits are used for \acp{id} supplementary meta data called Application Family Identifier.

\begin{table}
    \centering
    \includegraphics[width=\textwidth]{./figs/02-state-of-the-art/table_pcbits.pdf}
    \caption{\ac{pc} assignments from \ac{epc} \ac{uhf} \ac{gen2} Air Interface Protocol~\cite{UHFGen2Tag}} 
    \label{tab:pcbits}
\end{table}

\subsubsection{TID Memory (Bank $10$)}

\ac{tid} Memory (Bank $10$) holds chipset manufactured information and tag capability indicators. This memory bank is permalocked at the time of manufacture, being that it can not be changed.

There are misconceptions on the internet about the \ac{tid} "number".
The \ac{tid} Memory bank in contains two fields dedicated to manufacturer information called \ac{mdid} and \ac{tmn}.
\ac{mdid} encodes the Tag chip manufacturer \ac{id} which is attributed by GS1 as a unique identifier.
The \ac{tmn} is attributed by the manufacturer of the chip and describes the chipset number.

In the \ac{tds}, where the encoding of \ac{tid} is specified, it is referred has as \textbf{Short} Tag Identification.
This is because the \ac{tid} can be extended, which is them called \ac{xtid}.

\ac{xtid} is intended to provide more information about the capabilities of tags. It extends the \ac{tid} format by adding support for serialization and information about key features implemented by the tag~\cite{EPCTagData}.
Serialization is unique number generated by the tag manufacturer and can be used to uniquely identify one tag from another.
This identifies the tag itself, rather than item it is applied to.
The \ac{xtid} format became so common that people started referring to \ac{mdid}, \ac{tmn} and serialization has \emph{\ac{tid} number}.
The \ac{xtid} is specially useful in cases where the \ac{epc} isn’t serialized or is invalid.
An example of a \ac{tid} memory bank with \ac{xtid} can be seen in figure~\ref{fig:tid}.

\begin{figure}
    \centering
    \includegraphics[width=\textwidth]{./figs/02-state-of-the-art/tid.pdf}
    \caption{\ac{tid} Memory Bank of Monza R6-P Series used in this dissertation~\cite{TIDMemoryMaps}} 
    \label{fig:tid}
\end{figure}

\subsubsection{User Memory (Bank $11$)}

The User Memory Bank provides a variable size memory to store additional data related to the tagged object.
The bank implementation is optional and must be indicated in the \ac{umi} bit in the \ac{pc} Word.
It is frequently used to save information like temperature, maintenance logs, expiration dates and other data.

To ensure compatibility with other protocols, the first eight bits of the Bank shall contain a \ac{dsfid} as specified in ISO15962~\cite{isoISOIEC15962}.
This dissertation does not make use of the User Memory Bank whereby it will not be covered.
For further reference, GS1 presents a solution for encoding \ac{scm} data in the User Memory Bank in the \ac{tds}~\cite{EPCTagData}.

The access to the bank is made in blocks, through the \ac{gen2} Air Protocol. The arrangement of the data in the Bank is important as it can impact reading speeds.
In applications requiring fast reading speeds, User Memory Bank access speed can be a bottle neck when not implemented properly. (find citation for this)
GS1 provides a user memory encoder, a software that converts application data into a form suitable for storing into the user memory bank of a \ac{gen2} \ac{rfid} tag~\cite{marco.santos.diamondFAQ2020}.

\section{\ac{epc}} \label{sec:epc}

% Check page 24 of the GS1 EPCglobal architecture Framework pdf

\ac{epc} is a universal identifier used to identify every physical object anywhere in the world.
Differently from common \ac{uuid} identifiers, \ac{epc} has a set of collective terms for the identification code standardized by GS1~\cite{GS1GeneralSpecifications}. These terms convey context about the physical object in the encoding of the \ac{epc} itself.

\subsection{\ac{epc} in EPCGlobal Architecture Framework}

\acp{epc} are presented throughout the EPCGlobal Framework in various levels of abstraction. From low-level binary encoded, in \ac{gen2} \ac{rfid} tags, to business level applications.
The Framework presents seven representations~\footnote{binary, tag-encoding \ac{uri}, pure-identity \ac{uri}, legacy, legacy AI, element string and \ac{ons} hostname} for a single \ac{epc}. In general, only three are primarily used: \emph{Pure Identity}, \emph{Tag \ac{uri}} and \emph{Binary Encoding}.
Referring back to figure~\ref{fig:archstructure}, we can observe that the three representations are used in different contexts.

The \emph{Pure Identity \ac{epc} \ac{uri}} format is, as the name suggests, represented as an \ac{uri}.
GS1 uses the \ac{urn} scheme with the \mintinline{python}{epc} \ac{nid} registered for the EPCglobal \ac{epc} and related standards~\cite{meallingUniformResourceName}.
It is the most platform agnostic representation of an \ac{epc}, offering human readability and compatibility between heterogeneous systems. 

For components, like middlewares, requiring more information about the \ac{epc} memory bank of \ac{gen2} \ac{rfid} tags, the \ac{tds} provides the \emph{Tag \ac{uri}} scheme.
This representation maintains the \ac{uri} representation, changing the \ac{nss} from \mintinline{python}{id} to \mintinline{python}{tag}, and adding \textit{control information} used to guide the process of data capture from \ac{rfid} tags.
This scheme preserves information regarding the \ac{epc} Memory Bank in the \ac{urn} namespaces that are usually disregarded in business applications but necessary in middleware operations.
In other words, the \emph{\ac{epc} Tag \ac{uri}} is a text equivalent of the entire \ac{epc} memory bank contents.

\begin{figure}[!ht]
    \centering
    \includegraphics[width=0.7\textwidth]{./figs/02-state-of-the-art/SGTIN_First2encodings.jpg}
    \caption{Shit 1~\cite{SGTININFO}} 
    \label{fig:1111}
\end{figure}

The Binary Encoding of \ac{epc} contains a compressed encoding of the EPC and additional \textit{control information} in a compact binary form.
It will be made a in depth analysis of the encoding in section~\ref{sec:binencoding}.

\begin{figure}[!ht]
    \centering
    \includegraphics[width=\textwidth]{./figs/02-state-of-the-art/SGTIN_binaryconv2.jpg}
    \caption{Shit 2~\cite{SGTININFO}} 
    \label{fig:2222}
\end{figure}

\subsection{Relationship between EPCs and GS1 keys}

Before preceding to the encoding of \ac{epc} in \ac{gen2} \ac{rfid} tags, lets understand the underlying concepts of \ac{epc} schemes.

Previously, we mentioned that the \ac{epc} was designed to identify every physical object in the world.
The EPCGlobal Framework utilize this concept to its very extent.
A physical object in a \ac{scm} can be a broad term that does not provides much information regarding the object itself.
For that, the architecture defines \ac{epc} schemes and corresponding GS1 keys.
Each GS1 key denotes a class or grouping of physical objects. These classes encompass trade items, locations, assets, logistic units, transport groupings to say a few.
These GS1 keys add valuable information to \ac{scm} operations.
It allows intervening entities in the supply chain and logistics to retrieve context information about the tagged objects.
A deeper look in to GS1 keys and implementation guidelines is done in section~\ref{sec:identificationkeys}.

For each GS1 key there is a corresponding \ac{epc} scheme including both an \ac{epc} \ac{uri} and a binary encoding for use in \ac{rfid} tags.
Each \ac{epc} scheme and corresponding GS1 key can be seen in Appendix~\ref{tab:epcschemes}.

\subsection{SGTIN}

For the scope of this dissertation, the \ac{sgtin} scheme is the most important encoding scheme to look at.
The \ac{sgtin} encodes a \ac{gtin} plus a unique product or serial number.

\begin{figure}[!ht]
    \centering
    \includegraphics[width=0.8\textwidth]{./figs/02-state-of-the-art/SGTIN_UPC_Compare.jpg}
    \caption{Shit 3~\cite{SGTININFO}} 
    \label{fig:3333}
\end{figure}

The \ac{gtin} is used by companies to identify trade items~\cite{GS1GTINExecutive, GS1GTINManagement}. 
GS1 defines trade items as products or services that are priced, ordered or invoiced at any point in the supply chain.
\ac{gtin} can be used to identify types of products at any packaging level (e.g., consumer unit, inner pack, case, pallet~\footnote{even if a \ac{gtin} can be used to identify pallets, the use of \ac{sscc} is preferable~\cite{GS1KeysImplementation}}).

There are four \ac{gtin} formats: \ac{gtin}-$8$, \ac{gtin}-$12$, \ac{gtin}-$13$ and \ac{gtin}-$14$. In \ac{rfid} applications and general \ac{it} applications, it is used the uniform 14-digit format, \ac{gtin}-$14$. Is is the \ac{gtin} specified for \ac{gen2} \ac{rfid} \ac{sgtin} encoding.
Other \ac{gtin} formats are mainly used in barcodes in point-of-sale applications: in the \ac{us} is commonly used a \ac{gtin}-$12$ with UPC barcodes for single products and \ac{gtin}-$14$ with ITF-14 barcodes for product grouping. In contrast, in the rest of the world is commonly used \ac{gtin}-$8$ with \ac{ean}-8 barcodes for single products and \ac{gtin}-$13$ with \ac{ean}-13 for single products packaging configurations~\cite{BarcodeGS1General,GS1BarcodeChart}.

A \ac{gtin} is composed of three field: \emph{Company Prefix}, \emph{Item Reference} and check digit. All this fields are encoded in the different N-digit \ac{gtin} formats. 

GS1 Company Prefix is an identifier licensed by GS1 to identify a company globally. 
Nowadays, the registration of such is essential to companies whistling to enter digital marketplaces (e.g. Amazon~\cite{ListingRequirementsProduct}, Ebay) and big retail stores.
A Company Prefix does not uniquely identify a manufacturer and a company can register and own more than one~\cite{GS1EPCglobalArchitecture}. In certain circumstances Company Prefix can change.
When licensing a Company Prefix, there is an assessment and plan of unique product items a company shall produce. 
Depending on that, GS1 attributes a prefix with a length adjusted to company requirements. 
Companies requiring high quantities of unique item are given a shorter Company Prefix, accommodating more digits for item identification.
Also worth mentioning, the attribution of Company Prefix is made by one of the GS1 branches. Each branch has a prefix which uses to assign Company Prefixes (e.g.\ GS1 Portugal has $560$, GS1 Schweiz, Suisse and Svizzera have $760$-$769$)~\cite{anonymousGS1CompanyPrefix2014}.

Item References are assign by the managing entity of the product.
The item references must be concatenated with the Company Prefix to calculate the Check Digit, forming a \ac{gtin}.
From a \ac{gtin}, an \ac{sgtin} can be commissioned allowing to uniquely identification of a product within its product grouping.
Important to clarify, when a \ac{gtin} is stored in \ac{rfid} tags, as in \ac{sgtin} coding schemes, the \emph{Check Digit} presented in barcodes does not exist and must be dropped in barcode \ac{gtin} to \ac{epc} \ac{gtin} conversions.

\subsection{Binary Encoding} \label{sec:binencoding}

Lets now see how \acp{epc} are stored in \ac{gen2} \ac{rfid} tags.
We will focus on \ac{sgtin} encoding scheme, but the encoding method is analogous to other encoding schemes.

The binary encoding of an \ac{epc} consists of a fixed length header followed by a series of fields inherent to the encoding scheme like showed on \ac{sgtin}-$96$ Coding table~\ref{tab:sgtin96codingtable}.
For instance, lets take the \emph{Tag \ac{uri}}
\mintinline{python}{urn:epc:tag:sgtin-96:1.76300544.07470.2}
has example.
This \emph{Tag \ac{uri}} has \ac{sgtin}-$96$ encoding scheme, \emph{Filter value} of $``1"$, \emph{Company Prefix} $``76300544"$, \emph{Item Reference} $``7470"$ and serial $``2"$.

Following table~\ref{tab:encodingexample}, we first need to address the \ac{epc} header.
A snippet of the \ac{epc} header values can be seen in table~\ref{tab:epcheaders}. We observe that an \ac{sgtin} with $96$~bit coding scheme encodes a $8$-bit header with binary value of $``00110000"$.

Next, we need to encode the additional \textit{information} included in the \ac{epc} Memory Bank: the \emph{Filter} and \emph{Partition}. 

The \emph{Filter} encodes the packing level. An \ac{epc} \ac{sgtin} can be used to identify different levels of item packaging sharing the same \ac{gtin}. Differently from barcodes, which only encode the \ac{gtin}, an \ac{epc} with \emph{Filter} field allows a coherent \ac{gtin} across item packagings.
The filter value, has \textit{control information}, allows \ac{rfid} readers to select or deselect tags in the \emph{\ac{epc} \ac{uhf} \ac{gen2} Air Interface Protocol} corresponding to certain filter levels. This make it easier to read the desired tags in an environment where there may be other tags present (e.g.\ logistics companies only wanting to read and track Unit Loads like pallets).
Referencing table~\ref{tab:sgtinfiltervalues} with a , with a filter value of $``1"$, it encodes a \ac{pos} Trade Item with binary encoding of $``001"$.

The \emph{Partition} value encodes a pair of variable-length numeric fields referent to the \emph{Company Prefix} and \emph{Item Reference} memory partition. Previously we mentioned that \emph{Company Prefixes} can vary in length depending on company requirements for unique \emph{Item References}. 
In barcodes the \emph{Company Prefix} and \emph{Item Reference} are encoded together in a \ac{gtin} value. 
In \ac{gen2} tags, although there is fixed size memory shared between \emph{Company Prefix} and \emph{Item Reference}, there is also the \emph{Partition} value which specifies the distribution of that partition.
In the case of the example \emph{Tag \ac{uri}}, \emph{Company prefix} has $27$ bits / $8$ digits and \emph{Item reference} has $17$ bits / $5$ digits. Referring to table~\ref{tab:partitiontable} with the pair of variable-lengths, the \emph{Partition Value} is $``4"$ encoded has $``100"$.

The \emph{Company Prefix}, \emph{Item Reference} and \emph{serial} are encoded converting from decimal to binary and add leading zeros to fill all bits in each Logical Segment.

\begin{table}[]
    \centering
    \includegraphics[width=\textwidth]{./figs/02-state-of-the-art/table_codingtable.pdf}
    \caption{Coding Table of \ac{sgtin}-$96$~\cite{EPCTagData}} 
    \label{tab:sgtin96codingtable}
\end{table}

\begin{table}[]
    \begin{adjustwidth}{-0.00\textwidth}{0em}
    \begin{tabular}{@{}lllrl@{}}
        \toprule
        \textbf{Field}                                            & \textbf{Value (dec)} & \textbf{Value (bin)}                                                              & \textbf{\begin{tabular}[c]{@{}r@{}}Length \\ (bits)\end{tabular}} & \textbf{Notes}                                                                                                      \\ \midrule
        \ac{epc} Header                                                & 48                   & 00110000                                                                          & 8                                                                 & \ac{sgtin}-$96$ encoding                                                                                                 \\
        Filter Value                                              & 1                    & 001                                                                               & 3                                                                 & \ac{pos} item                                                                                                            \\
        Partition Value                                           & 4                    & 100                                                                               & 3                                                                 & Company Prefix has 8 digits                                                                                         \\
        \begin{tabular}[c]{@{}l@{}}Company \\ Prefix\end{tabular} & 76300544             & \begin{tabular}[c]{@{}l@{}}1001000110001000\\ 00100000000\end{tabular}            & 27                                                                &                                                                                                                     \\
        \begin{tabular}[c]{@{}l@{}}Item\\ Reference\end{tabular}  & 7470                 & 00001110100101110                                                                 & 17                                                                & \begin{tabular}[c]{@{}l@{}}Length depends on the partition\\ MSB zero fill\end{tabular}                             \\
        Serial                                                    & 2                    & \begin{tabular}[c]{@{}l@{}}0000000000000000\\ 0000000000000000000010\end{tabular} & 38                                                                & \begin{tabular}[c]{@{}l@{}}Serial is of fixed length.\\ Can be extended with \\ bigger EPC Memory Bank\end{tabular} \\ \bottomrule
        \end{tabular}
    \end{adjustwidth}
    \caption{\ac{sgtin}-$96$ binary encoding example of \mintinline{python}{urn:epc:tag:sgtin-96:1.76300544.07470.2} \emph{Tag \ac{uri}} retrieved from the Platform developed in this dissertation} 
    \label{tab:encodingexample}

\end{table}

\begin{table}[]
    \centering
    \includegraphics[width=\textwidth]{./figs/02-state-of-the-art/table_epcheaders.pdf}
    \caption{\ac{epc} headers snippet adapted from \ac{tds}~\cite{EPCTagData}} 
    \label{tab:epcheaders}
\end{table}

\begin{table}[]
    \centering
    \includegraphics[width=\textwidth]{./figs/02-state-of-the-art/table_sgtin_filtervalues.pdf}
    \caption{\ac{sgtin} Filter Value Table~\cite{EPCTagData}} 
    \label{tab:sgtinfiltervalues}
\end{table}

\begin{table}[]
    \centering
    \includegraphics[width=\textwidth]{./figs/02-state-of-the-art/table_partitionvalues.pdf}
    \caption{\ac{sgtin} Partition Table~\cite{EPCTagData}} 
    \label{tab:partitiontable}
\end{table}

\section{Low Level Reader Protocol (LLRP)}

The \acf{llrp} is specification for the network interface between \ac{rfid} Readers and Client applications~\cite{ImpinjLTKProgrammers}.

\subsection{Design Principles}

In some \ac{rfid} systems, there is a requirement for explicitly tune \ac{rfid} air protocols and the ability to control Readers that implement \ac{rfid} air protocol communications~\cite{LowLevelReader}.

Devices intended to operate \ac{rfid} Readers can vary from software applications, to middleware on local hardware, \acp{plc} and even on the cloud.

\ac{llrp} was design without requiring real-time interaction between the application software and Reader but with time-critical tasks at heart. 
The Reader application software passes operational rules to the Reader in non-real time.
The Reader then triggers and runs those operation rules to achieve time-critical requirements. 
The triggers can come from the application directly, from timers, \ac{gpi} hardware, or any other trigger defined by the Reader. 
This declarative operation method allows Readers to achieve peak performance without constraints caused by network or host latency.

\subsection{Connection Details}

\ac{llrp} is a binary protocol which runs over the \acs{tcp}/\acs{ip} internet transport protocols.
It is an asymmetric protocol where the \ac{llrp} client send commands to the Reader.

The protocol supports both reader and client initiated connections. By default, \ac{llrp} clients connect to \ac{tcp} port $5084$. In reader initiated connections, the Reader will actively try to establish a connection with the host application.
\ac{llrp} does not specify the behavior delivering data when the connection if broken. The reader used in this dissertation and others I've seen will continue to collect tag data and optionally deliver upon resumption of the connection.

Reader only allows one \ac{llrp} connection at any time. The \ac{tcp} connection between reader and client stays open till the client closes it or connection drops.

\subsection{Operation}

The primary function of the \ac{llrp} interface is to allow a client to finely tune the \ac{gen2} Tag Air Standard parameters and command the reader to inventory tags and otherwise access tags for read, write, lock, and kill.

For this, \ac{llrp} defines \acp{spec} which contain descriptions of 'what, how and when' the Reader perform determined operations.
\acp{spec} are triggered following boundary trigger information defined in it or by immediate triggers from applications, for a more imperative behavior.
A \ac{spec} is defined with parameters and fields. \acp{spec} are sent inside of messages, which are the unit of communication between client and Reader.
\ac{llrp} contains forty basic messages which range from commands, responses, events and a \textit{CUSTOM\_MESSAGE} for vender extensions. A list and description of every message can be found in Appendix~\ref{anx:llrpmessages}.

Inside messages, data is sent as fields or parameters.
Fields are individual data elements with a known format.
Parameters are named data elements that contain other parameters and/or fields, much like structures in programming languages.
When constructing \ac{llrp} messages, parameters can be optional, but field must be presented and within their valid range. Some tooling software provides user friendly features like inferring Reader capabilities and default values for unspecified fields.

\subsection{\acf{spec}}

There are two main \acp{spec} defined in \ac{llrp}, \textit{\acp{rospec}} and \textit{Access \acp{spec}}, which for major of the reader operations, encloses most of other \acp{spec}.
I will briefly go through some of the relevant ones required to understand this dissertation. Detailed information on important fields will be discussed later in this dissertation.

\subsubsection{\acf{rospec}}

\acp{rospec} control the operation of the Reader. It describes the inventory operations the Reader is instructed to perform.
A Reader only supports one \ac{rospec} at a time.

An example of an \textit{ADD\_ROSPEC} message in \ac{xml} representation can be seen in LISTING~\ref{code:rospec}. \ac{xml} is used in the context of many \ac{llrp} clients to provide human readability of the binary protocol.
Inside of the \textit{ADD\_ROSPEC} tag we can se an \ac{rospec}.
An \ac{rospec} contains the mandatory following elements: 

\begin{description}
    \item[ROSpecID] is an \ac{id} set by the client to uniquely identify the spec in subsequent commands.
    \item[Priority] must always be set to $0$.
    \item[CurrentState] describes the current state of the \ac{rospec}. When adding and \ac{rospec} the state must be set to \emph{Disable}.
    \item[ROBoundarySpec] is a parameter containing a description of the start and stop conditions for the \ac{spec}.
    \item[\ac{aispec}] contains the settings for the Reader antennas and \ac{gen2} Air protocol options. An \ac{rospec} can have one or more \acp{aispec} which are executed when the \ac{rospec} runs.
    \item[ROReportSpec] is optional and describes 'when' reports should be forward to the client and 'what' the report contains. If not defined, the Reader will use the current or default settings.
\end{description}
 

\begin{listing}
    \inputminted[linenos]{xml}{./code/sota/llrp_messages/ROSPEC.xml}
    \caption{\textit{ADD\_ROSPEC} message in \ac{xml} representation}
    \label{code:rospec}
\end{listing}


\subsubsection{Access \ac{spec}}

\subsection{Typical \ac{llrp} Exchange}

We have talked about \acp{spec}

\begin{figure}[!ht]
    \centering
    \includegraphics[width=0.7\textwidth]{./figs/02-state-of-the-art/llrpflow.pdf}
    \caption{Example of \ac{llrp} Application Flow~\cite{ImpinjLTKProgrammers}} 
    \label{fig:llrpflow}
\end{figure}



\section{Filtering \& Collection}

\section{Application Level Events (ALE)}

\section{Capture Application}

\section{EPCIS}

% ver sec 5.1.1 da EPCISGuidelinePdf

\subsection{Capture and Query Interfaces}

\subsection{Core Business Vocabulary}

\section{Practical Context}

% Nespresso supply-chain example and vision in the context of the EPCglobal framework, how it would work and advantages

\cleardoublepage