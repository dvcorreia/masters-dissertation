A gestão da cadeia de suprimentos evoluiu significativamente ao longo dos anos. Tornou-se mais digitalizada e automatizada, fundamentalmente mudando a forma como produtos são comercializados, distribuídos e administrados. Esforços têm sido feito para desenvolver e padronizar tecnologias, de modo a promover uma cadeia de suprimentos mais otimizada, criar visibilidade,  controlar os níveis de estoque, e prever flutuações do mercado.
Ainda assim, os esforços parecem terminar na integração da cadeia de suprimentos com pontos de venda.

Esta dissertação apresenta um estudo sobre estantes inteligentes com RFID, aplicadas a pontos de venda e almoxarifado, implementadas com tecnologias e padrões utilizados na cadeia de suprimentos, para atingir uma integração completada na rede de abastecimento.
A dissertação começa por apresentar RFID e seus princípios,
seguido por uma visão geral da coleção de tecnologias da arquitetura da \emph{EPCGlobal}, atualmente usada na cadeia de suprimentos, e o estado da arte das estantes e prateleiras inteligentes baseadas em RFID.

Em seguida, apresenta uma análise das opções de implementação, configurações de hardware e propostas de implementação para produto real.
Conclui apresentando uma implementação e testes do sistema de gerenciamento de inventário, usando componentes genéricos de UHF RFID disponíveis no mercado e software de código aberto, integrado em estantes típicas de alumínio/MDF para armazéns.